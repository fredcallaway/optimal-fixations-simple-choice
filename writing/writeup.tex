\documentclass[12pt,a4paperpaper,]{article}

\usepackage{lmodern}
\usepackage{amssymb,amsmath}

% use upquote if available, for straight quotes in verbatim environments
\IfFileExists{upquote.sty}{\usepackage{upquote}}{}
% use microtype if available
\IfFileExists{microtype.sty}{%
\usepackage{microtype}
\UseMicrotypeSet[protrusion]{basicmath} % disable protrusion for tt fonts
}{}

\usepackage{geometry}
\geometry{
    top=1in,
    bottom=1in,
    left=1in,
    right=1in,
    headheight=3ex,
    headsep=3ex
}
\usepackage{sidecap}
\usepackage{subfigure}

\usepackage[unicode=true]{hyperref}
\hypersetup{
            pdftitle={Attention in value-based choice as optimal sequential sampling},
            pdfborder={0 0 0},
            breaklinks=true}
\urlstyle{same}  % don't use monospace font for urls

% \usepackage[natbibapa]{apacite}
\usepackage{natbib}
% \bibliographystyle{apacite}
\bibliographystyle{apalike}
% \citationstyle{apacite}
% \bibpunct{[}{]}{,}{n}{}{;}


\usepackage{graphicx,grffile}
\makeatletter
\def\maxwidth{\ifdim\Gin@nat@width>\linewidth\linewidth\else\Gin@nat@width\fi}
\def\maxheight{\ifdim\Gin@nat@height>\textheight\textheight\else\Gin@nat@height\fi}
\makeatother
% Scale images if necessary, so that they will not overflow the page
% margins by default, and it is still possible to overwrite the defaults
% using explicit options in \includegraphics[width, height, ...]{}
\setkeys{Gin}{width=\maxwidth,height=\maxheight,keepaspectratio}
\IfFileExists{parskip.sty}{%
\usepackage{parskip}
}{% else
\setlength{\parindent}{0pt}
\setlength{\parskip}{6pt plus 2pt minus 1pt}
}
\setlength{\emergencystretch}{3em}  % prevent overfull lines
\providecommand{\tightlist}{%
  \setlength{\itemsep}{0pt}\setlength{\parskip}{0pt}}
\setcounter{secnumdepth}{0}

% % Redefines (sub)paragraphs to behave more like sections
% \ifx\paragraph\undefined\else
% \let\oldparagraph\paragraph
% \renewcommand{\paragraph}[1]{\oldparagraph{#1}\mbox{}}
% \fi
% \ifx\subparagraph\undefined\else
% \let\oldsubparagraph\subparagraph
% \renewcommand{\subparagraph}[1]{\oldsubparagraph{#1}\mbox{}}
% \fi

% set default figure placement to htbp
\makeatletter
\def\fps@figure{htbp}
\makeatother


\usepackage{titlesec}

% % \titleformat*{\section}{\LARGE\bfseries\sffamily}
% % \titleformat*{\subsection}{\Large\bfseries\sffamily}
% % \titleformat*{\subsubsection}{\large\series}
% % \titleformat*{\paragraph}{\large\bfseries\sffamily}
% % \titleformat*{\subparagraph}{\large\bfseries\sffamily}

\titlespacing*{\section}{0pt}{3ex}{1ex}
\titlespacing*{\subsection}{0pt}{2ex}{0.5ex}
\titlespacing*{\subsubsection}{0pt}{1.5ex}{0.3ex}
\titlespacing*{\paragraph}{0pt}{1ex}{3ex}

%% Headers and footers:
%%
%% Center footer: page number
%% Right header:  nothing
%% Left header:   nothing
% \usepackage{lastpage}
% \newpagestyle{fancy}{
%   \setfoot{}{\thepage \ of \pageref*{LastPage}}{}
%   \sethead
%     { Attention in value-based choice as optimal sequential sampling }
%     {}
%     { Callaway }
%   \headrule
%   \setheadrule{0.3pt}
% }
% \pagestyle{fancy}  

\title{\vspace{-2em}Attention in value-based choice\\
       as optimal sequential sampling}
\author{Frederick Callaway \& Thomas L. Griffiths}

\date{\vspace{-1em}}

\begin{document}
\maketitle


\newcommand{\A}{\mathcal{A}}
\newcommand{\B}{\mathcal{B}}
\renewcommand{\S}{\mathcal{S}}
\newcommand{\D}{\mathcal{D}}
\newcommand{\R}{\mathbb{R}}
\newcommand{\Z}{\mathbb{Z}}
\newcommand{\expect}[2][]{\mathbb{E}_{#1} \left[ #2 \right]}
\newcommand{\nop}{\texttt{NOOP}}
\newcommand{\Normal}{\mathrm{Normal}}
\renewcommand{\vec}[1]{\mathbf{#1}}

Consider a diner at a new restaurant, perusing the menu and trying to decide what she wants to have for dinner. Under the standard model of economic decision making \citep{Kahneman1979,Rangel2008}, she would assign each item a value and then choose the one with maximal value. While her preferences might change day to day (leading to inconsistent choices), she always chooses the item that she likes best at the moment. Unfortunately, this idealization does not capture the experience many of us have. Instead, we undergo a difficult and sometimes lengthy process of weighing the options, initially being drawn to one entr\'ee before identifying a competitor, oscillating between them, failing to notice a desirable choice, and feeling pangs of regret at the sight of our companion's meal.

A great deal of work in psychology and neuroscience has attempted to better capture the process through which people make decisions. One key insight is that decision-making is a sequential process. This insight is formalized in the drift-diffusion model \citep{Ratcliff1978,Milosavljevic2010}, where the oscillation between mushroom risotto and pesto gnocchi is captured by the perturbations of a random walk. However such models fail to capture the important role that attention plays in the decision making process \citep{Orquin2013}. One robust empirical effect is that people are more likely to choose items that they look at more. This effect might be due to an attentional bias, in which attending to an item artificially inflates the value you assign to it. Such a bias is formalized in the attentional drift diffusion model (aDDM), in which the drift rate is shifted in favor of the fixated item \citep{Krajbich2010}. In addition to explaining this basic effect of attention on choice, the aDDM predicts subtler patterns. For example, last fixations are predicted to be shorter than first fixations, because the fixation is cut off when the decision variable crosses a barrier.

% eye-mind assumption (see Orquin review)

Despite these successes, several important questions have not been addressed by previous work. The vanilla drift-diffusion model has been shown to be equivalent to a sequential probability ratio test \citep{Bogacz2006,Bitzer2014}, providing a rational explanation for why people might use a DDM-like decision mechanism. However, there is no such rational justification for why people would bias their choices with their attention. Indeed, it is hard to imagine how such a bias could possibly improve the quality of choices. Nevertheless, there is some indication that people direct their attention in an adaptive manner: When presented with three or more options, people tend to look more at higher valued items, which in turn makes them more likely to choose those items \citep{Krajbich2011}. Although there is an intuition for why people would direct their attention in this manner, there is no formal model of how people allocate their attention in such cases, much less a rational model of how people \emph{should} allocate their attention.

Here, we attempt to provide a rational model of attention allocation in decision making within the framework of resource-rational analysis \citep{griffiths15}. We begin by formalizing the attention allocation problem as a process of sequential sampling and posterior updating. By endogenizing attention within this model (i.e.~allowing an agent to select which item to consider at each time point), we define a sequential decision problem \emph{for how to make a single decision}. We formalize this problem as metalevel Markov decision process \citep{Hay2012}, and identify an approximately optimal solution using a recently developed reinforcement learning technique \citep{callaway2018learning}. With this approach, we can make precise predictions about both how attention should affect choice as well as how attention should be allocated, all within a single rational model. We discuss the logic of two key predictions regarding the correlations between value, attention and choice. Finally, we test these predictions by applying the model to the dataset of \citet{Krajbich2011}, in which participants made ternary choices while under eye-tracking, finding that the model captures many of the key qualitative patterns in the human data.


\section{Computational model}

Before formally describing the model, we first provide some intuition. We treat decision making as an iterative process of sampling and inference. The decision maker (or \emph{agent}) is presented with a set of items, each of which has some true unknown value. In order to determine which item to choose, the agent can generate noisy samples of each item's utility, each sample providing a small amount of information about the utility of a single item, but also having a small cost. The agent continuously integrates information from these samples, thus developing an increasingly precise and accurate belief about each item's value.

The role of attention in this model is to select which item is sampled at each time step. Thus, we view the problem of how to allocate attention in decision making as a form of information search or active learning. Importantly, the agent cannot simply decide to allocate attention to the highest value item because she does not know the true values. Rather, she must decide which item to attend to based on her current estimations of each item's value. She may choose, for example, to focus her attention on items that appear to have high value, or perhaps items whose value is still highly uncertain. However, we do not specify how expected value and uncertainty should be traded off (in contrast to e.g.~upper confidence bound algorithms). Rather, we assume that the agent allocates attention in an (approximately) optimal way, maximizing the quality of the final decision and minimizing the cost incurred by the decision making process.

In addition to deciding what information to gather (i.e.~what item to attend to), the agent must decide when to stop gathering information. We assume that this decision is also made optimally: The agent stops stops gathering information and makes a decision when the expected increase in decision quality falls below the expected cost of gathering more information. As the sampling process progresses, the agent becomes increasingly confident in her estimates. As a result, the potential benefit of each new sample diminishes, eventually falling below the cost of gathering more information. In some cases, however, the agent quickly identifies a choice with very high value, and does not wait to attain high certainty in the exact value of each item before making a choice. In every case, the decision to make a choice is made by comparing the expected value of new information to its cost.

Our model is based on theoretical work in artificial intelligence, specifically the field of metareasoning \citep{Hay2012}, where provably optimal sampling strategies have been identified for a narrow set of problems. It is thus interesting, if not surprising, that the model bears some resemblance to the evidence accumulation models more familiar to psychologists, in particular the drift diffusion model (DDM). As in the DDM, decisions in our model arise from a process in which evidence for each item is accumulated over time, being tracked by internal ``decision variables'' that represent the estimated value of each item. However, our model contrasts with the DDM in that both expected values and uncertainty estimates are explicitly represented. This allows the decision variables to evolve according to optimal Bayesian inference.%
  \footnote{Only the binary, hypothesis-testing version of the DDM (e.g. as applied to perceptual decision making) has been shown to implement Bayesian inference. This is in contrast to the multi-alternative, value-based decisions we model here.}
Furthermore, we do not posit an explicit stopping rule such as a threshold, but instead use an expected value computation to determine when to terminate the decision making procedure. Finally, unlike any evidence accumulation models of which we are aware, we propose that the information is not gathered either uniformly for all items, or by a stochastic (exogenous) attentional process; instead, information is selected by a near-optimal meta-controller.

Having provided some intuition, we now present a formal model of attention allocation for value-based choice. We model the problem as a \emph{metalevel} Markov decision process, and discuss how we can find a near-optimal solution to the problem using a specially designed reinforcement learning algorithm.

\subsection{Attention allocation as a Metalevel Markov decision process}

A metalevel Markov decision process (meta-MDP) is a formalism developed in the artificial intelligence literature to describe the problem of how to allocate computational resources to best trade off between decision quality and computational cost. The key insight lies in viewing computation as a sequential process; an algorithm typically executes many individual operations to accomplish an ultimate goal, and the value of each of those operations cannot be determined without reference to all (or at least some) of the other operations in the sequence. Given this observation, it is natural to model computation as a Markov decision process because it is the standard formalism for modeling sequential decision problems.

A meta-MDP is formally identical to a standard MDP; it is defined by a set of states, a set of actions, a transition function, and a reward function. It is distinct from a standard MDP only in its interpretation and the way in which it is derived. In a meta-MDP the states correspond to the agent's beliefs, the actions correspond to computations (or cognitive operations), the transition function describes how computations update the agent's beliefs, and the reward function describes both the cost of computation and also the utility of the item that is ultimately chosen.%
  \footnote{Readers familiar with the partially observable Markov decision process (POMDP) literature will observe that the metalevel MDP is similar to the belief-MDP representation of a POMDP, with the addition of replacing physical actions with a single operation that takes the optimal choice given the current belief.}
We now define each of these elements.


\paragraph{Beliefs}
The agent's beliefs are described by a set of Gaussian distributions, each of which is the agent's posterior distribution for the utility of one item. We denote the posterior utility distribution for item $i$ at time point $t$ as $U_i(t) \sim \Normal(\mu_i(t), \lambda_i(t)^{-1})$. The belief at time step $t$, $b(t)$, can thus be encoded by two vectors giving the mean, $\boldsymbol{\mu}(t)$, and precision, $\boldsymbol{\lambda}(t)$, of the estimate of each item's value. The state space of the meta-MDP is thus $\R^k \times (0,\infty)^k$.

\paragraph{Computations}
The actions of the meta-MDP correspond to the computations (or cognitive operations) that the agent can perform. Although many different kinds of computations are likely to play a role in even simple decision, we consider a highly reduced set with one computation for each item, $\{c_1, c_2, \dots c_k \}$. Each computation draws a sample from an observation distribution for a single item, $\Normal(u_i, \sigma^2)$, with mean equal to the item's true utility and variance, $\sigma^2$, being a free parameter of the model. The posterior distribution for that item's utility, $U_i$, is then updated according to Bayesian inference. In expectation, this brings the MAP utility estimate (i.e. $\mu_i$) closer to the true utility $u_i$; however, due to sampling noise, a single computation may actually increase the difference between the estimate and the ground truth. This operation can be interpreted agnostically as ``considering item $i$''.%
  \footnote{One could also give more weight to the sampling mechanism as capturing the kind of computation brains naturally perform \citep{sanborn16}.}
Additionally, we define a special computation, $\bot$, which indicates that the agent terminates the decision making process and makes the best choice given her current beliefs, choosing an item according to $\arg\max_i \mu_i(t)$.

\paragraph{Transition function}
The metalevel transition function describes how computations affect beliefs. We assume that the transition function is governed by Bayesian inference. Specifically, the posterior distribution of the item considered at each time step is updated according to a sample drawn from the corresponding item's observation distribution. Let $c$ denote the item considered at time step $t$. The transition dynamics are then defined by the following equations:

\begin{equation}
\begin{aligned}
  o(t) &\sim \Normal(u_{c}, \sigma^2) \\
  \lambda_c(t+1) &= \lambda_c(t) + \sigma^{-2}  \\
  \mu_{c}(t+1) &= \frac{\sigma^{-2} o(t) + \lambda_{c}(t) \mu_{c}(t)}{\lambda_{c}(t+1)}  \\
  \lambda_i(t+1) &= \lambda_i(t) \text{ for } i \neq c  \\
  \mu_i(t+1) &= \mu_i(t) \text{ for } i \neq c  \\
\end{aligned}
.
\end{equation}

\paragraph{Reward function}
The metalevel reward function captures both the costs of computation and also the quality of the decision that is ultimately made. The costs are defined
%
\begin{equation}
R(b_t, c_t) = -\samplecost - \mathbf{1}(c_t \neq c_{t-1}) \switchcost
\,,
\end{equation}
%
where the first term captures the cost of sampling and the second captures an additional switching cost for considering a different item than the one considered on the previous time step. To maintain the Markov property, we simply add the previous computation, $c_{t-1}$, as an auxiliary state variable. 
% ${\text{cost}_\text{sample}}$ and ${\text{cost}_\text{switch}}$ are free parameters.
The metalevel reward function captures decision quality by assigning a reward to the termination action, $\bot$. When this action is selected, the agent chooses the best item given her current beliefs, $i^*(t) = \arg\max_i \mu_i(t)$. 
The reward for terminating is thus $R(b_t, \bot) = u_{i^*(t)}$.


\subsection{Optimal attention allocation as an optimal metalevel policy}
Having formalized the problem of attention allocation for decision making as a metalevel Markov decision process, the problem of identifying an optimal attentional strategy is reduced to the problem of identifying the optimal policy for an MDP. The policy of a meta-MDP is a function that returns a computation (or distribution over computations) to take in a given belief state. A policy is optimal if it maximizes total expected metalevel reward. Equivalently, it always chooses computations that maximize the \emph{value of computation} (VOC), which is defined as the expected benefit of executing additional computations rather than making a decision immediately.
% Formally, $\text{VOC}(b, c)$ is defined as the expected sum of all future metalevel rewards minus the reward of terminating in belief state $b$, given that computation $c$ is executed in belief state $b$ and assuming that all future computations are chosen optimally
The VOC is nearly equivalent to the state-action value function (often denoted $Q$) of a standard MDP: $\text{VOC}(b, c) = Q(b, c) - R(b, \bot)$. However, it has the advantage of separating the potential value of \emph{additional} information from the value of the information that has already been acquired. Thus, efforts to solve metareasoning problems have typically been devoted to approximating the VOC.

Because the belief state space is continuous, standard dynamic programming or tabular reinforcement learning techniques cannot be applied; function approximation methods are necessary. We employ a recently developed policy search method that was developed specifically for meta-MDPs \citep{callaway2018learning}. The method approximates the VOC as a weighted sum of features. The value of information to be gained from further computation is captured by the expected improvement in decision quality that would be gained from (1) a single computation, (2) learning everything about one item, and (3) learning everything about all items. The cost of future computations is approximated by a single intercept term. With a sum-to-one constraint on the first three feature weights, this results in a three-dimensional policy, the weights of which are optimized to maximize metalevel return. The method has been shown to find near-optimal policies in a similar meta-MDP to the one above (with Bernoulli observations instead of Gaussian).

% For any belief-computation pair, the true value of information is necessarily a convex combination of these features, although the weights of each component may differ for different belief-computation pairs. The method simply assumes that the weigahts are constant for all belief-computation pairs and learns the 

 % uses hand-designed \emph{value of information} features. 


 % for the value of executing a single computation, the value of acquiring perfect knowledge about one item, and the value of acquiring perfect knowledge about all items. The first feature is lower bound on the $VOC$ and the third is an upper bound. The VOC is then approximated as a convex combination of these features, with an additional term to capture expected future costs. We optimize the policy's three parameters (two free parameters for the convex combination and one for cost) to maximize the expected total metalevel reward that the policy receives on random decision problems.%
  % \footnote{Add more detail here. See Callaway et al. (2018) for details in the mean time.}

\subsection{Model fitting}
Although, the parameters of the policy are optimized (reflecting our assumption that people allocate attention rationally), the meta-MDP itself has three free parameters that must be fit to human data.

\begin{enumerate}
  \item $\sigma$ is the standard deviation of the observation distribution that each sample is drawn from. It controls how quickly and reliably the estimates of each item's utility converges to to the true value. Higher values generally result in slower and less accurate decisions.
  \item $\samplecost$ is the cost of taking a sample. It controls how many samples the optimized policy tends to take. Higher values generally result in faster and less accurate decisions.
  \item $\switchcost$ is the additional cost of sampling from a different item than that fixated on the previous time step. Higher values generally result in fewer fixations (changes of attention), and slightly decrease decision time and accuracy.
\end{enumerate}

Ideally, we would optimize these parameters via maximum likelihood estimation. However, there is no analytical expression for the likelihood, and it is quite challenging to approximate it by simulation due to the high dimensionality of the data (in particular, the time series of fixations). Thus, we fit the model by minimizing the mean absolute error on summary statistics for a given dataset and model simulations.

To connect the model to behavioral data, we require three additional parameters. The first, $\sampletime$, determines how long (in seconds) each sample takes, allowing us to predict fixation and reaction times from the number of samples the model takes. Rather than fitting this parameter, we set it to match the model's mean predicted reaction time (across all trials) to the true mean reaction time. The second and third parameters, $\mu_0$ and $\sigma_0$ allow us to translate the ratings given by participants (on a -10 to +10) scale into values that the model assumes to be unit-normally distributed. The model values are set to $u = \frac{r - \mu_0}{\sigma_0}$ where $r$ is a rating between -10 and 10. Although these parameters could be fit, we simply validate the model's assumption of unit-normality by setting $\mu_0$ and $\sigma_0$ to the empirical mean and standard deviation of ratings for all items shown in the experiment.

% However, doing so would be equivalent to assuming that people immediately adapt their priors (and as a result, their decision making strategy) to capture the statistics of the experimental task. In particular, because choice trials were restricted to items that were rated positively, we might expect people's priors to be systematically pessimistic, generally being pleasantly surprised by the high values of the items they see. This would correspond to having $\mu_0$ less than the empirical rating mean. If no task-adjustment takes place and people interpret a rating of 0 as average, $\mu_0$ should be set to 0. In practice, we have found (so far) that how we handle these two parameters doesn't matter very much; they seem to trade off with the other parameters somehow so we can get similar quality fits to the human data with any reasonable setting.


% fixation_bias.pdf
% gaze_cascade.pdf
% last_fix_bias.pdf


\section{Results}

\begin{figure}[t!]
  \centering
  \subfigure[]{
    \includegraphics[width=.45\textwidth]{figs/value_choice.pdf}
    \label{fig:value_choice}
  }
  \subfigure[]{
       \includegraphics[width=.45\textwidth]{figs/difference_time.pdf}
       \label{fig:difference_time}
  }
  \caption{Basic psychometrics.
    (a) Probability of choosing an item as a function of it's relative value. Relative value is defined as the item's rating minus the mean rating for all options in the given trial.
    (b) Total fixation time as a function of the relative value of the item with maximal value (an indicator of inverse difficulty).
    The dashed red lines indicate the model simulation. Bars denote 95\% confidence intervals estimated by bootstrapping.
  }
  \label{fig:fixate_on_best}
\end{figure}

To test the model's ability to capture the structure of human attention in value-based choice, we apply the model to the dataset of Krajbich \& Rangel (2011), in which participants made ternary choices between snack items. Participants were not under time pressure and visual attention was recorded with eye tracking. For each of the 2966 trials in the dataset, we simulate ten model runs, using the ratings the participants provided in the first stage of the experiment as the true utilities $u_i$ of each item. We fit model parameters by minimizing the total mean absolute error on all seven plots shown in this paper. The model makes two key qualitative predictions about the relationships between attention, value, and choice that are relatively insensitive to the specific parameter settings. First, we find that attention is generally directed towards items that have higher true utilities. Second, we find that more-attended items are more likely to be chosen, even holding value constant. We now describe and interpret each of these findings in greater detail.

\subsubsection{Attention is directed to high-value items}

\begin{figure}[t!]
  \centering
  \subfigure[]{
    \includegraphics[width=.45\textwidth]{figs/value_bias.pdf}
    \label{fig:value_bias}
  }
  \subfigure[]{
       \includegraphics[width=.45\textwidth]{figs/fixate_on_best.pdf}
       \label{fig:fixate_on_best}
  }
  \caption{Attention is directed to high-value items.
    (a) Proportion of total fixation time spent on a given item as a function of its relative value.
    (b) Probability of fixating on the item with maximal value as a function of the time since trial onset. Trials which lasted less than 2 seconds are excluded from this plot so that all plotted points are based on the same set of trials (to avoid confounding time with difficulty).
    The dashed red lines indicate the model simulation. Bars denote 95\% confidence intervals estimated by bootstrapping.
  }
  \label{fig:attend_to_value}
\end{figure}

\begin{figure}[t!]
  \centering
  \subfigure[]{
    \includegraphics[width=.45\textwidth]{figs/gaze_cascade.pdf}
    \label{fig:value_bias}
  }
  \subfigure[]{
       \includegraphics[width=.45\textwidth]{figs/last_fix_bias.pdf}
       \label{fig:fixate_on_best}
  }
  \caption{The chosen item is often fixated last.
    (a) The probability of fixating on the chosen item as a function of fixation number, aligned to the final fixation.
    (b) The probability of choosing the last fixated item as a function of it's value.
    The dashed red lines indicate the model simulation. Bars denote 95\% confidence intervals estimated by bootstrapping.
  }
  \label{fig:choose_last}
\end{figure}

Our model predicts that a rational agent will allocate more attention to items that she believes are valuable. As shown in Figure~\ref{fig:attend_to_value}a, humans appear to do the same. To see why this is rational, consider a case in which there are two items with similarly high value and one item with much lower value. The agent can quickly determine that the low value item is not the best one, and thus allocates most of her attention to the two high-value items to determine which one to choose. In the reverse case, in which there is only one high value item, the agent quickly identifies it and has no reason to determine which of the similarly low valued items is superior; attention is roughly equally divided among the three options. Thus, in net, positively valued items receive more attention.

Importantly, the true item values do not directly influence attention; this effect is mediated by the internal value estimates that the agent constructs over the course of a decision. In the initial stages of a decision, the agent's value estimates are uncertain and noisy, with a weak dependence on the true values. Although she attends most to the items she \emph{believes} are most valuable, she is likely to be mistaken in her belief and thus frequently attends to low-value items. As the decision progresses, the agent refines her belief, and the estimated values that are biasing attention become closer to the true values. As a result, we expect that the tendency to attend to high value items will increase over the course of the decision. Indeed, as shown in Figure~\ref{fig:attend_to_value}b, this pattern holds in both the human data and the model simulations.

The direction of attention to items with high estimated value may also explain the tendency for the last fixation to be on the chosen item (Figure~\ref{fig:choose_last}). In the attentional drift diffusion model, this effect is explained by an interaction between the attentional bias and the decision threshold. When an item is fixated, the drift rate is biased in that item's favor; thus, when a decision threshold is crossed, it is likely to be in favor of the currently fixated item. Rational attention allocation provides an alternative explanation. The optimal agent tends to focus on whatever item she currently believes is best, which is also the item she chooses when she terminates computation. Thus, when the agent decides to terminate computation and make a choice, she will most likely choose the item she is currently fixating. Interestingly, the decision to fixate \textit{away from} an item is in indicator that she believes the item is less promising than another, perhaps explaining the dip in the probability of fixating the chosen item in the second-to-last fixation (Figure~\ref{fig:choose_last}a). Furthermore, because both attentional allocation and the ultimate choice are based on the agent's \textit{estimates} of value, the ``last fixation bias'' holds even when conditioning on true value (Figure~\ref{fig:choose_last}b).

Importantly, this prediction only holds for decisions between three or more items. Indeed, it has been shown that attention should be exactly evenly divided in the two alternative case \citep{Fudenberg2018}, and our policy optimization method recovers this behavior. Accordingly, we see a relatively weak effect of value on attention in the two-alternative case \citep{Krajbich2010}. This small effect might be due to people applying a heuristic attention-allocation policy that is explicitly biased towards high-value items. Such a strategy could closely approximate optimal attention allocation in the more common multi-alternative case, but result in slightly suboptimal behavior in the two-alternative case. Exploring heuristic strategies that can explain this phenomenon is an important direction for future research.


\subsubsection{More attended items are more likely to be chosen}

\begin{figure}[tb]
  \centering
  \includegraphics[]{figs/fixation_bias.pdf}
  \caption{Attention and choice.
    Items that are fixated on longer are more likely to be chosen. Relative fixation time is defined as the amount of time spent fixating on the given item minus the mean fixation time for all items in the trial.
    The dashed red lines indicate the model simulation. Bars denote 95\% confidence intervals estimated by bootstrapping.
  }
  \label{fig:fixation_bias}
\end{figure}


% \begin{verbatim} to be fixating on 
%                   Figure 0

% Mechanism #1: attention -> estimated value -> choice

% Mechanism #2: attention <- estimated value -> choice
% \end{verbatim}


In our model, the more an agent attends to an item, the more likely she is to ultimately choose it (Figure~\ref{fig:fixation_bias}). This prediction is also made by the aDDM, a direct result of the assumption that the drift rate is positively biased towards the attended item. However, our model makes the same prediction without any biases in the evidence accumulation process

There are two mechanisms through which the effect can emerge in our model. The first mechanism is through a mismatch between the prior distribution and the true distribution from which items are drawn. Value estimates in our model are based on a combination of a prior and likelihood, where the weight of the likelihood increases with computation time. Thus, for an item with true value above the prior mean, the estimated value tends to increase with computation time, and the converse holds for items with true value below the prior mean \citep{Armel2008}. If the prior is unbiased, then the positive and negative effects of attention on valuation wash out. However, if the prior is biased, i.e. if choice items are systematically better or worse than expected, attention has a corresponding systematic effect on valuation.

The second mechanism by which the apparent attentional bias may emerge depends on the rational allocation of attention. As described in the previous section, the rational model predicts that (in the case of three or more items), the agent is more likely to attend to items that she believes are valuable. At the same time, she is also more likely to choose items that she believes are valuable. Thus, a high estimated value makes the agent more likely to both attend to and choose an item. As a result, attention and choice are correlated by a common-cause structure. This contrasts to the mechanism described above in which attention has a causal effect on choice, as mediated by estimated value. Importantly, both choice and attention allocation depend on estimated value, not the true item values; thus, the correlation is not broken by conditioning on true value.

For the results presented here, we normalized the ratings such that they have zero-mean and unit-variance. Thus, the model's prior is unbiased and only the second mechanism can apply. Empirically, we find that if we allow the prior mean to vary (i.e. allowing the model to fit a biased prior), the best-fitting prior is the empirical one. Thus, it appears that a biased prior is not necessary (or even helpful) in fitting this particular dataset. However, this could be an artifact of our fitting procedure, which is not based on maximum likelihood estimation. Thus, further work is needed to determine whether the observed correlation between attention and choice is truly best explained by the common-cause mechanism alone.

We can distinguish between these two mechanisms by comparing two- vs. three-alternative choice for positive vs. negative items. The biased-prior mechanism is insensitive to the number of alternatives, and will predict a positive (negative) effect of attention when items are generally positive (negative). In contrast, the biased-attention mechanism is insensitive to average value, and will predict a positive effect of attention when there are more than two alternatives, and no effect of attention when there are just two alternatives.


\section{Discussion}
We presented a rational model of attention allocation in value-based decision making that formalizes attention allocation as a sequential sampling problem. Here, we focused on two key predictions of the model: attention is directed to high-value items and more attended items are more likely to be chosen. However, because the model predicts full sequences of fixations, we can look for additional model predictions in the same way that we analyze rich and complex human data, through exploratory data analysis. Preliminary analyses on these lines have revealed, for example, that the model predicts longer reaction times for trials in which all items are relatively bad, perhaps reflecting a form of loss attention \citep{yechiam2013losses}. Further investigating the model's predictions is a key direction for future work.

In the results presented here, we normalized the ratings such that the model's prior is unbiased, ruling out the biased-prior explanation for the correlation between attention and choice. Thus, Figure~\ref{fig:fixation_bias} seems to suggest that this correlation (at least as it appears in this dataset) can be explained entirely by a common-cause structure in which high estimated values lead to both more attention and greater choice probability. However, experimental manipulations of attention suggest that there is in fact a causal effect of attention on choice \citep{Tavares2017}. In further work, we plan to apply our model to this and other datasets to further elucidate the role attention plays in decision making.



% \setlength{\bibleftmargin}{.125in}
% \setlength{\bibindent}{-\bibleftmargin}
% \renewcommand*{\bibfont}{\footnotesize}
\bibliography{references}


\end{document}