\documentclass[12pt,a4paperpaper,]{article}

\usepackage{lmodern}
\usepackage{amssymb,amsmath}

% use upquote if available, for straight quotes in verbatim environments
\IfFileExists{upquote.sty}{\usepackage{upquote}}{}
% use microtype if available
\IfFileExists{microtype.sty}{%
\usepackage{microtype}
\UseMicrotypeSet[protrusion]{basicmath} % disable protrusion for tt fonts
}{}

\usepackage{geometry}
\geometry{
    top=1in,
    bottom=1in,
    left=1in,
    right=1in,
    headheight=3ex,
    headsep=3ex
}
\usepackage{sidecap}
\usepackage[unicode=true]{hyperref}
\hypersetup{
            pdftitle={Attention allocation in multi-alternative choice},
            pdfauthor={Fred Callaway},
            pdfborder={0 0 0},
            breaklinks=true}
\urlstyle{same}  % don't use monospace font for urls

\usepackage{natbib}
\bibpunct{[}{]}{,}{n}{}{;}
\bibliographystyle{apa}


\usepackage{graphicx,grffile}
\makeatletter
\def\maxwidth{\ifdim\Gin@nat@width>\linewidth\linewidth\else\Gin@nat@width\fi}
\def\maxheight{\ifdim\Gin@nat@height>\textheight\textheight\else\Gin@nat@height\fi}
\makeatother
% Scale images if necessary, so that they will not overflow the page
% margins by default, and it is still possible to overwrite the defaults
% using explicit options in \includegraphics[width, height, ...]{}
\setkeys{Gin}{width=\maxwidth,height=\maxheight,keepaspectratio}
\IfFileExists{parskip.sty}{%
\usepackage{parskip}
}{% else
\setlength{\parindent}{0pt}
\setlength{\parskip}{6pt plus 2pt minus 1pt}
}
\setlength{\emergencystretch}{3em}  % prevent overfull lines
\providecommand{\tightlist}{%
  \setlength{\itemsep}{0pt}\setlength{\parskip}{0pt}}
\setcounter{secnumdepth}{0}

% Redefines (sub)paragraphs to behave more like sections
\ifx\paragraph\undefined\else
\let\oldparagraph\paragraph
\renewcommand{\paragraph}[1]{\oldparagraph{#1}\mbox{}}
\fi
\ifx\subparagraph\undefined\else
\let\oldsubparagraph\subparagraph
\renewcommand{\subparagraph}[1]{\oldsubparagraph{#1}\mbox{}}
\fi

% set default figure placement to htbp
\makeatletter
\def\fps@figure{htbp}
\makeatother


\usepackage{titlesec}

% \titleformat*{\section}{\LARGE\bfseries\sffamily}
% \titleformat*{\subsection}{\Large\bfseries\sffamily}
% \titleformat*{\subsubsection}{\large\series}
% \titleformat*{\paragraph}{\large\bfseries\sffamily}
% \titleformat*{\subparagraph}{\large\bfseries\sffamily}

\titlespacing*{\section}
{0pt}{3ex}{1ex}
\titlespacing*{\subsection}
{0pt}{2ex}{0.5ex}
\titlespacing*{\subsubsection}
{0pt}{1.5ex}{0.3ex}

%% Headers and footers:
%%
%% Center footer: page number
%% Right header:  nothing
%% Left header:   nothing
\usepackage{lastpage}
\newpagestyle{fancy}{
  \setfoot{}{\thepage \ of \pageref*{LastPage}}{}
  \sethead
    { Attention allocation in multi-alternative choice }
    {}
    { Callaway }
  \headrule
  \setheadrule{0.3pt}
}
\pagestyle{fancy}

\title{\vspace{-2em}Attention allocation in multi-alternative choice}
\author{\vspace{-1em}Fred Callaway}
\date{\vspace{-1em}}

\begin{document}
\maketitle


\newcommand{\A}{\mathcal{A}}
\newcommand{\B}{\mathcal{B}}
\renewcommand{\S}{\mathcal{S}}
\newcommand{\D}{\mathcal{D}}
\newcommand{\R}{\mathbb{R}}
\newcommand{\Z}{\mathbb{Z}}
\newcommand{\expect}[2][]{\mathbb{E}_{#1} \left[ #2 \right]}
\newcommand{\nop}{\texttt{NOOP}}
\newcommand{\Normal}{\mathrm{Normal}}
\renewcommand{\vec}[1]{\mathbf{#1}}

Here we consider the problem of a decision maker who must choose between
one of several goods. The decision maker has limited attentional
capabilities and is under some time pressure (perhaps only due to
opportunity cost). Given these constraints, how should the agent divide
her attention among the competing options? Here we derive a near-optimal
solution to this problem and compare to human fixation patterns.

\subsection{Problem statement}\label{problem-statement}

An agent must choose between \(k\) goods, each having some true utility
\(u_i\). The agent does not have knowledge of these exact values, but
she can draw samples from \(k\) Normal distributions centered on these
true utilities. At some point she chooses one of the items and receives
a payout equal to the true utiliy of the chosen item.

We assume sampling has a cost, which may be due to explicit time cost,
implicit opportunity cost, internal cognitive cost, or some combination
of the three. We additionally assume that \emph{changing} the focus of
attention is costly; that is, it is more costly to sample from a
distribution that was not sampled on the last time step. Because we are
interested in modeling eye tracking data, we refer to the item that was
most recently sampled from as the \emph{fixated} item; we refer to
changing the focus of attention, i.e.~sampling from a different
distribution as a \emph{saccade}.

\subsection{POMDP model}\label{pomdp-model}

We can model this problem as a partially observable Markov decision
process (POMDP) which defines a set of states, actions, and observations
as well as functions that determine how action and state determine
observations and future states. In our model, state defines the true
(unknown) utility and sampling precision of each item as well as the
(known) attentional state of the agent i.e.~which item is fixated.
Actions can be attentional (saccading to a new item) or physical
(choosing an item). Observations are noisy estimates of the fixated
item's utility. Rewards capture the cost of time/attention, the
additional cost of saccades, and the utility of the item that is
ultimately chosen. Formally, we define a POMDP
\((\S, \A, T, R, \Omega, O)\) where

\begin{itemize}
\tightlist
\item
  \(\S = \R^k \times \{1, \dots, k\}\) is the state space. The state is
  broken up into the true utility of each item \(\vec{u}\), and the
  current fixation \(f\).
\item
  hello there
\item
  \(\A = \{ \nop, f_1, \dots, f_k , c_1, \dots, c_k \}\) is the action
  space, where \(\nop\) has no effect, \(f_i\) saccades to item \(i\),
  and \(c_i\) chooses item \(i\), ending the episode.
\item
  \(T: \S \times \A \rightarrow \S\) is the deterministic transition
  function that updates only the fixation portion of state based on
  saccade actions.
\item
  \(R: \S \times \A \rightarrow \R\) is the reward function that gives a
  constant negative reward at each time step, plus an additional
  negative reward for making saccades. For the choice actions, it gives
  a reward equal to the utility of the chosen item.
\item
  \(\Omega = \R\) is the set of possible observations, which are samples
  of an item's utility.
\item
  \(O: \S \times \A \times \Omega \rightarrow [0, 1]\) is the
  observation function that gives the probability of drawing a utility
  sample given the true utility of the fixated item.
  \(O(s, o) = \Normal(o; u_{f}, \sigma)\) where \(f\) is the fixated
  item and \(\sigma\) is a free parameter that determines how noisy the
  samples are.
\end{itemize}

\subsection{Optimal solution}\label{optimal-solution}

Following Kaelbling et al. (1998), we break down the problem into two
parts: a state estimator and a policy. The state estimator maintains a
belief (i.e.~a distribution over states) based on the sequence of
actions and observations. The policy selects the action to take at each
time step given the current belief. By combining these two parts, we
create an agent that optimally selects fixations and choices given the
full history of previous observations.

\subsubsection{State estimator}\label{state-estimator}

For ease of exposition, we focus on the belief over item values, noting
that the currently fixated item is simply the target of the most recent
fixation action. The belief over item values \(\vec{u}\) is a
multivariate Gaussian with mean vector \(\vec{\mu}\) and diagonal
precision matrix \(\Sigma = \text{diag}(\vec{\lambda})\). Because the
observation \(o\) is only informative about the currently fixated item
(indicated by subscript \(f\) below), only the belief about the fixated
item changes at each time step. We derive this update by Bayesian
inference {[}cite Murphy?{]}, resulting in

\[
\begin{aligned}
\lambda_f(t+1) &= \lambda_f(t) + \sigma^{-2}  \\
\mu_f(t+1) &= \frac{\sigma^{-2} o + \lambda_f(t) \mu_f(t)}{\lambda_f(t+1)}  \\
\lambda_i(t+1) &= \lambda_i(t) \text{ for } i \neq f  \\
\mu_i(t+1) &= \mu_i(t) \text{ for } i \neq f  \\
\end{aligned}
\]

The belief is initialized to the prior distribution over \(\vec{u}\).
For now, we assume that the agent knows the true distribution from which
utilities are sampled from, which are standard-normal distributed; thus,
we have \(\vec{\mu}(0) = 0\) and \(\vec{\lambda}(0) = 1\).

\subsubsection{Policy}\label{policy}

The policy makes two kinds of actions: saccades and choices. We can
immediately simplify the problem by reducing the set of choices to a
single action, \(\bot\), which selects the item that has maximal
expected value given the current belief, \(\arg\max_i \mu_i(t)\). With
this reduction, the problem becomes a \emph{metareasoning} problem (Hay
et al. 2012): at each time step the policy must decide whether or not to
gather more information and which (if any) item to gather information
about. Such problems are generally impossible to solve exactly due to
the infinite (continuous) space of possible beliefs. To address this
difficulty, Callaway et al. (2018) proposed a reinforcement learning
method for identifying metareasoning policies. They found that their
method, Bayesian Metalevel Policy Search (BMPS), found near-optimal
policies for a bandit-like metareasoning problem with similar structure
to the present problem. Thus, we apply their method and treat the
identified policy as ``optimal''.




\end{document}